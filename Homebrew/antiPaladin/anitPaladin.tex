\documentclass[22pt,a4paper]{article}
\usepackage[utf8]{inputenc}
\usepackage{amsmath}
\usepackage{amsfonts}
\usepackage{amssymb}
\usepackage{color}
\usepackage{fontspec}
\usepackage{textpos}
\usepackage{multicol}
\usepackage[singlelinecheck=false]{caption}
\usepackage[table,xcdraw]{xcolor}
\usepackage{wallpaper}
\captionsetup[table]{labelformat=empty}
\usepackage{multirow}
\definecolor{title}{RGB}{94,18,0}
\definecolor{sepcol}{RGB}{186,159,11}
\usepackage[top=1in, left=1in, right=1in]{geometry}
\ULCornerWallPaper{1}{pageBackground.jpg}
\newfontfamily\bookmania{Bookmania}
\setmainfont{Bookmania}[Scale=0.9]
\newcommand{\HRule}[2]{\par
  \vspace*{\dimexpr-\parskip-\baselineskip+#2}
  \begingroup
  	\color{sepcol}
  	\noindent\rule{\linewidth}{#1}\par
  \endgroup  
  \vspace*{\dimexpr-\parskip-.5\baselineskip+#2}}
\begin{document}
\begin{multicols}{2}
\noindent{\fontsize{22pt}{22pt}\textcolor{title}{\bookmania \textbf{\textsc{Anti-Paladin}}}}\\
A human in dull black armor pounces forward, bringing his greatsword down on a troll. The wound festers and the blood runs like water.\\
\indent A tiefling whips a halberd around, slicing into a man’s leg. She quickly mutters an in incantation, and the man gasps and falls. \\
\indent A half-elf raises his shield and shouts to the skies. Unholy light falls around his enemies, making them easy prey.\\
\indent An anti-paladin steals divine magic to fulfill their goals, and makes vows against the gods to enhance their power further. Through dark rituals and incantations, anti-paladins gain the ability to fight the gods and undo their work. Anti-paladins come in many forms, yet their driving force remains the same: bringing justice to the divine.\\
\\
{\fontsize{18pt}{18pt}\textcolor{title}{\textsc{Judge of the Divine}}}
\HRule{1pt}{8pt}
An anti-paladin of any sort stands against the gods. They exist to bring a god to some form of justice, and steal from their power to do so. However, anti-paladins don’t necessarily want the world to fall into chaos. Most anti-paladins make a vow against a single god or goddess, and do everything in their power to undo their work while leaving the work of other gods alone. While different anti-paladins may focus on different skills in order to best fulfill their goals, anti-paladins all make similar vows against the god or gods that they wish to bring justice to.\\
\indent Anti-paladins train for a large part of their life in order to fulfill their goal, learning various fighting skills and training with a multitude of weapons and armors. However, the power they steal from the gods is much greater than any physical training. Causing sickness and disease, raising undead, and creating dangerous storms are the least of their abilities.\\
\\
{\fontsize{18pt}{18pt}\textcolor{title}{\textsc{Life of an Anti-Paladin}}}
\HRule{1pt}{8pt}
Anti-Paladins rarely settle in a single location for long, as the meddling of the gods happens in many places. They often take on the role of an adventurer, searching for evidence of the gods’ touch and erasing it. Exploring ancient ruins or eliminating a cult can help an anti-paladin discover the ways in which the gods have touched the world, and provide useful information for them in their quest against the divine.\\
\end{multicols}
\newpage
\ThisCenterWallPaper{1}{pageBackgroundTable.jpg}
\begin{textblock}{16}(0.2,-0.3)
\textsc{The Anti-Paladin}
\end{textblock}
\begin{picture}(30,300)(-1,-168)
\label{my-label}
\begin{tabular}{m{3mm}cclccccc}
&                        &                                     &                                            & \multicolumn{5}{c}{-Spell Slots per Spell Level-} \\
&\multirow{-2}{*}{Level} & \multirow{-2}{*}{\begin{tabular}[x]{@{}c@{}}Proficiency\\Bonus\end{tabular}} & \multirow{-2}{*}{Features}                 & 1st      & 2nd      & 3rd     & 4th     & 5th \\
\rowcolor[HTML]{9AFF99} 
&1st                     & +2                                  & Divine Sense, Vile Hands                   & -        & -        & -       & -       & -      \\
&2nd                     & +2                                  & Fighting Style, Spellcasting, Unholy Smite & 2        & -        & -       & -       & -      \\
\rowcolor[HTML]{9AFF99} 
&3rd                     & +2                                  & Blightbringer, Heinous Vow                 & 3        & -        & -       & -       & -      \\
&4th                     & +2                                  & Ability Score Improvement                  & 3        & -        & -       & -       & -      \\
\rowcolor[HTML]{9AFF99} 
&5th                     & +3                                  & Extra Attack                               & 4        & 2        & -       & -       & -      \\
&6th                     & +3                                  & Unsettling Aura                            & 4        & 2        & -       & -       & -      \\
\rowcolor[HTML]{9AFF99} 
&7th                     & +3                                  & Heinous Vow Feature                        & 4        & 3        & -       & -       & -      \\
&8th                     & +3                                  & Ability Score Improvement                  & 4        & 3        & -       & -       & -      \\
\rowcolor[HTML]{9AFF99} 
&9th                     & +4                                  & -                                          & 4        & 3        & 2       & -       & -      \\
&10th                    & +4                                  & Aura of Intimidation                       & 4        & 3        & 2       & -       & -      \\
\rowcolor[HTML]{9AFF99} 
&11th                    & +4                                  & Improved Unholy Smire                      & 4        & 3        & 3       & -       & -      \\
&12th                    & +4                                  & Ability Score Improvement                  & 4        & 3        & 3       & -       & -      \\
\rowcolor[HTML]{9AFF99} 
&13th                    & +5                                  & -                                          & 4        & 3        & 3       & 1       & -      \\
&14th                    & +5                                  & Corrupting Touch                           & 4        & 3        & 3       & 1       & -      \\
\rowcolor[HTML]{9AFF99} 
&15th                    & +5                                  & Heinous Vow Touch                          & 4        & 3        & 3       & 2       & -       \\
&16th                    & +5                                  & Ability Score Improvement                  & 4        & 3        & 3       & 2       & -       \\
\rowcolor[HTML]{9AFF99} 
&17th                    & +6                                  & -                                          & 4        & 3        & 3       & 3       & 1       \\
&18th                    & +6                                  & Aura Improvements                          & 4        & 3        & 3       & 3       & 1       \\
\rowcolor[HTML]{9AFF99} 
&19th                    & +6                                  & Ability Score Improvement                  & 4        & 3        & 3       & 3       & 2       \\
&20th                    & +6                                  & Heinous Vow Feature                        & 4        & 3        & 3       & 3       & 2      
\end{tabular}
\end{picture}
\begin{multicols*}{2}
{\fontsize{18pt}{18pt}\textcolor{title}{\textsc{Creating an Anti-Paladin}}}
\HRule{1pt}{8pt}
The most important aspect of an Anti-paladin is the Heinous Vow they make at third level. What drove them to hate a god or gods? Why do they seek vengeance? What kind of justice does your character wish to bring the god or gods to? Are you honorable, and wish for the gods to be judged in the eyes of mortals for their actions? Are you driven by revenge, desiring only the death of a god or gods to equalize the horrendous actions? Make sure to read the Heinous Vows section in preparation. \\
\indent Appendix B of the Player’s Handbook lists many gods that your character may have taken a vow against. Work with your DM to select the proper god or gods that pertain to your campaign. \\
\indent What caused you to take a vow against the gods? Did they destroy your village or clan, cause the death of your family, poison a river, or simply interfere where they didn’t belong? Perhaps you were raised in an environment where the hatred of the gods was encouraged? \\
\indent Despite their quest against the gods, anti-paladins aren’t necessarily evil or chaotic in alignment. Their alignment may depend largely on what god they have vowed against, or on how they intend to bring the god to justice. However, anti-paladins are rarely seen as good creatures.\\
\\
{\fontsize{12pt}{12pt}\textcolor{title}{\textsc{Quick Build}}}\\
You can quickly make an anti-paladin by following these suggestions. First, Strength should be your highest ability score, followed by Charisma. Second, choose the Outlander background.\\
\\
{\fontsize{18pt}{18pt}\textcolor{title}{\textsc{Class Features}}}\\
As an anti-paladin, you gain the following class features. \\
\\
{\fontsize{12pt}{12pt}\textcolor{title}{\textsc{Hit Points}}}\\
\textbf{Hit Dice: }1d10 per anti-paladin level\\
\textbf{Hit Points at 1st Level: }10 + your Constitution modifier\\
\textbf{Hit Points at Higher Levels: }1d10(or 6) + your Constitution modifier per anti-paladin level after 1st\\
\\
{\fontsize{12pt}{12pt}\textcolor{title}{\textsc{Proficiencies}}}\\
\textbf{Armor: }All armor, shield\\
\textbf{Weapons: }Simple weapons, martial weapons\\
\textbf{Tools: }None\\
\\
\textbf{Saving Throws: }Strength, Wisdom\\
\textbf{Skills: }Choose two from Athletics, Deception, Insight, Intimidation, Persuasion, Religion\\
\\
{\fontsize{12pt}{12pt}\textcolor{title}{\textsc{Equipment}}}\\
You start with the following equipment, in addition to the equipment granted by your background:
\begin{itemize}
\setlength\itemsep{-6pt}
\item (a) a martial weapon and a shield or (b) two martial weapons
\item (a) five javelins or (b) any simple melee weapon
\item (a) a priest's pack or (b) an explorer's pack
\item Chain mail and an unholy symbol
\end{itemize}
\newpage
\noindent{\fontsize{14pt}{14pt}\textcolor{title}{\textsc{Divine Sense}}}
\HRule{1pt}{8pt}
The dark presence of evil and the bright existence of good register on your senses like a foul odor, or a sweet treat. As an action, you can open your awareness to detect such forces. Until the end of your next turn, you know the location of any celestial, fiend, or undead within 60 ft. of you that is not behind total cover. You know the type (celestial, fiend, undead) of any being whose presence you sense, but not its identity (the vampire Dr. Acula, for instance). Within the same radius, you also detect the presence of any place or object that has been consecrated or defiled, as with the hallow spell. \\
\indent You can use this feature a number of times equal to 1 + your charisma modifier. When you finish a long rest, you regain all expended uses. \\
\\
{\fontsize{14pt}{14pt}\textcolor{title}{\textsc{Vile Hands}}}
\HRule{1pt}{8pt}
\noindent Using the stolen power of the gods, you can absorb the life energy of those around you. Dealing melee weapon damage or necrotic damage builds up a healing pool equal to the damage dealt. The maximum total hit points in the pool are equal to your anti-paladin level x 5.\\
\indent As an action, you can heal yourself or an undead ally you touch for an amount equal to the hit points in the pool. When you use this ability, the entire pool is expended. Hit points from the pool that exceed the target’s maximum health after healing are lost. The total amount of hit points you are able to expend in between long rests is limited to your anti-paladin level x 10.\\
\indent Alternatively, you can spend ten points from the pool to inflict a condition to a target that you can touch. Make a touch attack using your spell attack modifier. If the attack succeeds, the target makes a Constitution save equal to your spell save DC. On a failed save, it is inflicted with a condition from the following table. Roll a d100 and apply the resulting condition. If the condition cannot be applied to the target, use the condition above it on the table (as if the die had rolled lower) until a condition is applicable. The condition lasts for one minute. The conditions are outlined in “Table A: Conditions.”\\
\\
\textsc{Table: Random Condition}\\
\begin{tabular}{ll}
Roll              & Condition                \\
\rowcolor[HTML]{B0E1AF} 
1-10              & Invisible                \\
11-20             & Blinded                  \\
\rowcolor[HTML]{B0E1AF} 
21-30             & Poisoned                 \\
31-40             & Paralyzed                \\
\rowcolor[HTML]{B0E1AF} 
41-50             & Deafened                 \\
51-60             & Stunned                  \\
\rowcolor[HTML]{B0E1AF} 
61-70             & Frightened               \\
71-80             & Incapacitated            \\
\rowcolor[HTML]{B0E1AF} 
81-90             & Charmed                  \\
91-99             & Unconscious              \\
\rowcolor[HTML]{B0E1AF} 
100               & Petrified               
\end{tabular}\\
\vfill
\columnbreak
\noindent{\fontsize{14pt}{14pt}\textcolor{title}{\textsc{Fighting Style}}}
\HRule{1pt}{8pt}
At 2nd level, you adopt a style of fighting as your specialty choose one of the following options. You can't take a Fighting Style option more than once, even if you later get to choose again.\\
\\
{\fontsize{12pt}{12pt}\textcolor{title}{\textsc{Defense}}}\\
While you are wearing armor, you gain a +1 bonus to AC.\\
\\
{\fontsize{12pt}{12pt}\textcolor{title}{\textsc{Dueling}}}\\
When you are wielding a melee weapon in one hand and no other weapons, you gain a +2 bonus to damage rolls with that weapon.\\
\\
{\fontsize{12pt}{12pt}\textcolor{title}{\textsc{Great Weapon Fighting}}}\\
When you roll a 1 or 2 on a damage dire for an attack you make with a melee weapon that you are wielding with two hands, you can reroll the die and must tuse the new roll. The weapon must have the two-handed or versatile property for you to gain this benefit.\\
\\
{\fontsize{12pt}{12pt}\textcolor{title}{\textsc{Protection}}}\\
When a creature you can see attacks a target other than you that is within 5 feet of you, you can use your reaction to impose disadvantage on the attack roll.\\
You must be wielding a shield.\\
\\
{\fontsize{14pt}{14pt}\textcolor{title}{\textsc{Spellcasting}}}
\HRule{1pt}{8pt}
By 2nd level, you have learned to steal divine magic from the gods. See chapter ten for general rules of spellcasting, and see the anti-paladin spell list near the end of this article for what spells you can cast.\\
\\
{\fontsize{12pt}{12pt}\textcolor{title}{\textsc{Preparing and Casting Spells}}}\\
The anti-paladin table shows how many spell slots you have to cast your spells. To cast one of your anti-paladin spells of 1st level or higher, you must expend a slot of the spell’s level or higher. You regain all expended spell slots when you finish a long rest.\\
\indent You prepare the list of anti-paladin spells that are available for you to cast, choosing from the anti-paladin spell list. When you do so, choose a number of anti-paladin spells equal to your charisma modifier + half your anti-paladin level, rounded down (minimum one spell). The spells must be for a level for which you have spell slots. \\
\indent For example, if you are a fifth level anti-paladin, you have four 1st level spell slots and two 2nd level spell slots. With a Charisma of 14, your list of prepared spells can include four spells of 1st or 2nd level, in any combination. If you prepare the 1st level spell inflict wounds, you can cast it using a 1st level or 2nd level spell slot. Casting the spell doesn’t remove it from your list of prepared spells.
\indent You can change your list of prepared spells when you finish a long rest.\\
\\
{\fontsize{12pt}{12pt}\textcolor{title}{\textsc{Spellcasting Ability}}}\\ 
Charisma is your spellcasting ability for your anti-paladin spells, since their power derives from the strength of your convictions. You use your charisma whenever a spell refers to your spellcasting ability. In addition, you use your Charisma modifier when setting the saving throw DC for an anti-paladin spell you cast and when making an attack roll with one. \\
\begin{center}
\textbf{Spell Save DC} = 8 + your proficiency bonus + your Charisma modifier\\
\textbf{Spell Attack Modifier} = your proficiency bonus + your Charisma modifier
\end{center}
{\fontsize{12pt}{12pt}\textcolor{title}{\textsc{Spellcasting Focus}}}\\
You can use your unholy symbol (similar to a holy symbol in Chapter 5, “Equipment”) as a spellcasting focus for your paladin spells. \\
\\
{\fontsize{14pt}{14pt}\textcolor{title}{\textsc{Unholy Smite}}}
\HRule{1pt}{8pt}
Starting at 2nd level, when you hit a creature with a melee weapon attack, you can expend one spell slot to deal necrotic damage to the target, in addition to the weapon’s damage. The extra damage is 2d8 for a 1st level spell slot, plus 1d8 for each spell level higher than 1st, to a maximum of 5d8. The damage increases by 1d8 if the target is celestial.\\
\\
{\fontsize{14pt}{14pt}\textcolor{title}{\textsc{Blight Bringer}}}
\HRule{1pt}{8pt}
By 3rd level, your unholy ability to call forth diseases and plagues make you immune to disease. \\
\\
{\fontsize{14pt}{14pt}\textcolor{title}{\textsc{Heinous Vow}}}
\HRule{1pt}{8pt}
When you reach 3rd level, you make a vow that binds you as an anti-paladin forever. Up to this time, you have been in a preparatory stage, committed to the path but not yet sworn to it. Now you choose your Heinous Vow, three of which are detailed at the end of the class description (Vow of Balance, Vow of Death, Vow of Retribution).\\
\indent Your choice grants you features at 3rd level and again at 7th, 15th, and 20th level. Those features include Vow spells and the Twist Divinity feature.\\
\\
{\fontsize{12pt}{12pt}\textcolor{title}{\textsc{Vow Spells}}}\\
Each vow has a list of associated spells. You gain access to these spells at the levels specified in the oath description. Once you gain access to a Vow spell, you always have it prepared. Vow spells don’t count against the number of spells you prepare each day.\\
\indent If you gain a Vow spell that doesn’t appear on the anti-paladin spell list, the spell is nonetheless an anti-paladin spell for you.\\
\\
{\fontsize{12pt}{12pt}\textcolor{title}{\textsc{Twist Divinity}}}\\
Your vow allows you to alter divine energy stolen from the gods to create magical effects. Each Twist Divinity option provided by your oath explains how to use it.\\
\indent When you use your Twist Divinity, you choose which option to use. You must then finish a short or long rest to use your Twist Divinity again. 
Some Twist Divinity features require saving throws. When you use such an effect from this class, the DC equals your anti-paladin spell save DC.\\
\\
{\fontsize{14pt}{14pt}\textcolor{title}{\textsc{Ability Score Improvement}}}
\HRule{1pt}{8pt}
When you reach 4th level, and again at 8th, 12th, and 16th, and 19th level, you can increase one ability score of your choice by 2, or you can increase two ability scores of your choice by 1. As normal, you can’t increase an ability score above 20 using this feature.\\
\\
{\fontsize{14pt}{14pt}\textcolor{title}{\textsc{Extra Attack}}}
\HRule{1pt}{8pt}
Beginning at 5th level, you can attack twice, instead of once, when you take the attack action on your turn.\\
\\
{\fontsize{14pt}{14pt}\textcolor{title}{\textsc{Unsettling Aura}}}
\HRule{1pt}{8pt}
Starting at 6th level, whenever an enemy creature within 10 feet of you must make a saving thro, the creature's roll is reduced by an amount equal to your Charisma modifier (minimum bonus +1). You must be conscious for the effect to activate.\\
\indent At 18th level, the range of this aura increases to 30 feet.\\
\\
{\fontsize{14pt}{14pt}\textcolor{title}{\textsc{Aura of Intimidation}}}
\HRule{1pt}{8pt}
Starting at 10th level, when an enemy walks into or though your aura range of 10 feet, you can choose to frighten that creature. The target makes a Wisdom save against your spell save DC. If they fail, they are frightened for 30 seconds. Every turn following, the creature can make a Wisdom save at the end of their turn to end the effect, with the same spell save DC as before. Attacking the target will also end the effect.\\
\indent Any creature that has passed the initial Wisdom save or any creature that was affected by the aura and broke free of its effects by any means is immune to the effects of this aura for 24 hours.\\
\indent At 18th level, the range of this aura increases to 30 feet.\\
\\
{\fontsize{14pt}{14pt}\textcolor{title}{\textsc{Improved Unholy Smite}}}
\HRule{1pt}{8pt}
By 11th level, you have corrupted the touch of the gods enough that your melee weapon strikes carry unholy power with them. Whenever you hit a creature with a melee weapon, that creature takes an extra 1d8 necrotic damage. If you also use your Unholy Smire with an attack, you add this damage to the extra damage of your Unholy Smite.\\
\\
{\fontsize{14pt}{14pt}\textcolor{title}{\textsc{Corrupting Touch}}}
\HRule{1pt}{8pt}
Beginning at 14th level, you can use your action to end one spell on yourself or on one willing creature that you touch.\\
\indent You can use this feature a number of times equal to your Charisma modifier(a minimum of once). You regain all expended uses when you finish a long rest.\\
\\
{\fontsize{18pt}{18pt}\textcolor{title}{\textsc{Heinous Vows}}}\\
Becoming an anti-paladin involves making a vow that solidifies the anti-paladin’s form of justice he shall serve to the divine. The final vow, taken when he or she reaches 3rd level, is the apex of all the anti-paladin’s training. Some characters with this class don’t consider themselves true anti-paladins until they have reached 3rd level and made their vow. For others, they consider themselves on the path of an anti-paladin long before taking their vow, and make this vow to solidify their goals.\\
\\
{\fontsize{14pt}{14pt}\textcolor{title}{\textsc{Vow of Balance}}}
\HRule{1pt}{8pt}
The Vow of Balance binds an anti-paladin to the task of removing the touch of the gods in the event that they interfere too much. An anti-paladin who has sworn a Vow of Balance may have no particular hate for the gods but feels strongly that their interference in the world may be unnecessary and is therefore causing imbalance. Many who take this vow are neutral in the overall balance of the universe, and may also lean on good. However an anti-paladin’s intentions or actions while following a Vow of Balance can range throughout the entire alignment spectrum.\\
\\
{\fontsize{12pt}{12pt}\textcolor{title}{\textsc{Tenets of Balance}}}\\
\indent\textbf{View their Work: }Locate the work of the gods\\
\indent\textbf{Judge their Work: }Determine if their work is necessary\\
\indent\textbf{Balance their Work: }Remove their creations if they do no belong\\
\\
{\fontsize{12pt}{12pt}\textcolor{title}{\textsc{Vow Spells}}}\\
You gain vow spells at the anti-paladin level specified.\\
\\
\begin{tabular}{cl}
\begin{tabular}[c]{@{}c@{}}Anti-Paladin\\ Level\end{tabular} & \multicolumn{1}{c}{Spells}    \\
\rowcolor[HTML]{B0E1AF} 
3rd                & Detect Magic, Bane            \\
11-20              & Blur, Calm Emotions           \\
\rowcolor[HTML]{B0E1AF} 
21-30              & Conjure Barrage, Counterspell \\
31-40              & Staggering Smite, Stone Shape \\
\rowcolor[HTML]{B0E1AF} 
41-50              & Conjure Volley, Mislead      
\end{tabular}\\
\\
{\fontsize{12pt}{12pt}\textcolor{title}{\textsc{Twist Divinity}}}\\
When you take this vow at 3rd level, you gain access to the following two Twist Divinity options.\\
\indent\textbf{Neutralize: }As an action, you can alter the positive and negative nergies of the world around you, expending your Twist Divinity. Within a 10 ft. radius of your position, if a roll was made with advantage or disadvantage, you can neutralize the roll. The neutralized roll will be rolled as if there was no advantage or disadvantage applied to it. This does not affect advantage or disadvantage grant by magical means or racial traits, such as a Halfling's "Brave" trait, or the spell "True Strike."\\
\indent If the event that was neutralized was a singular event, such as neutralizing the advantage granted by a surprise round, the effect ends. If the event is repeatable, such as neutralizing the advantage gained by a single target attacking a prone ally, the effect may last up to one minute.\\
\\
\indent\textbf{Soil Divinity: }Upon calculating spell damage that is not necrotic damage, you can expend your Twist Divinity to change all of the initial damage that the spell will deal to necrotic damage. If additional damage is dealt to a creature on later turns, that damage will be of the original damage type of the spell. Spell effects, such as flammable objects catching on fire or light objects being knocked back, still occur as normal.\\
\\
{\fontsize{12pt}{12pt}\textcolor{title}{\textsc{Antimagic Aura}}}\\
By 7th level, your control of positice and negative nergies allows you to alter the powers of magic around you. You and alliws within 10 feet of you have resistance to damage from spells.\\
\indent At 18th leve, the range of this aura is extended to 30 feet.\\
\\
{\fontsize{12pt}{12pt}\textcolor{title}{\textsc{Neutralizing Aura}}}\\
By 15th level, your control of the positive and negative energies is reaching the point of mastery. Within 10 feet of you, if a roll is made with advantage or disadvantage, you may attempt to neutralize the roll. Roll 1 d20 if the roll is a 16 or higher, treat the advantageous or disadvantageous rolls as if it had been neutralized by your Twist Divinity feature.\\
\indent This aura has no effect on advantage or disadvantage granted by magic or racial traits.
\indent At 18th level, the range of this aura is extended to 30 feet.\\
\\
{\fontsize{12pt}{12pt}\textcolor{title}{\textsc{Balancing Act}}}\\
By 20th level, you have gained true understanding of the balance of the universe, and how to change it to best suit you. using your action, you can tip the balance of the world in your favor. For one minute, the following effects occur:\\
\begin{itemize}
\setlength\itemsep{-6pt}
\item All spell damage dealt by you is necrotic damage
\item Attack rolls and saving throws by you have advantage
\item All attack that you are the target of have disadvantage
\item You are immune to magical charm effects.
\end{itemize}
Once you use this feature, you can't use it again until you finish a long rest.
\\
\\
{\fontsize{14pt}{14pt}\textcolor{title}{\textsc{Vow of Death}}}
\HRule{1pt}{8pt}
Often called a Black Knight or a Death Bringer, an anti-paladin who has taken the Vow of Death has sworn to slay a god and remove all of their touch from the world. Despite being associated with storms necromancy, and diseases, an anti-paladin who has taken a Vow of Death does not necessarily exist to spread chaos and evil. However, it is unlikely for an anti-paladin to follow a lawful or a good lifestyle if they wish to weaken a god enough to kill. An anti-paladin who has taken the Vow of Death against a god or gods should speak to their DM to properly include role playing features and combat sessions into the campaign.\\
\\
{\fontsize{12pt}{12pt}\textcolor{title}{\textsc{Tenets of Death}}}\\
\indent\textbf{Locate: }Locate their work, and destroy it\\
\indent\textbf{Weaken: }Reduce their power by any means\\
\indent\textbf{Kill: }Remove them from the world\\
\\
{\fontsize{12pt}{12pt}\textcolor{title}{\textsc{Vow Spells}}}\\
You gain vow spells at the anti-paladin level specified.\\
\\
\begin{tabular}{cl}
\begin{tabular}[c]{@{}c@{}}Anti-Paladin\\ Level\end{tabular} & \multicolumn{1}{c}{Spells}     \\
\rowcolor[HTML]{B0E1AF} 
3rd                & False Life, Inflict Wounds     \\
11-20              & Darkness, Blindness/Deafness   \\
\rowcolor[HTML]{B0E1AF} 
21-30              & Animate Dead, Elemental Weapon \\
31-40              & Blight, Fire Shield            \\
\rowcolor[HTML]{B0E1AF} 
41-50              & Cloudkill, Contagion          
\end{tabular}\\
\\
{\fontsize{12pt}{12pt}\textcolor{title}{\textsc{Twist Divinity}}}\\
When you take this vow at 3rd level, you gain access to the following two Twist Divinity options.\\
\indent\textbf{Curse Wound: }Following a successful attack action you can expend your Twist Divinity to curse the creature that you have dealt damage too. The wound becomes cursed, drawing attacks to it. For the next minute, your attacks against the cursed target have advantage. \\
\\
\indent\textbf{Diseases Blade: }As an action, you can expend your Twist Divinity to coat your melee weapon with a poison that lasts for one minute. Damage dealt by this weapon (including damage dealt by your unholy smite) cannot be healed until the creature takes a short or long rest. Until the coating of the blade runs out, the creature suffers from the “poisoned” condition outlined in “Appendix A: Conditions.”\\
\\
{\fontsize{12pt}{12pt}\textcolor{title}{\textsc{Relentless Assault}}}\\
At 7th level, upon killing an enemy with a melee weapon attack, you may use a bonus action to move up to half of your movement speed and you may make an additional melee weapon attack. This can only occur once per turn. \\
\\
{\fontsize{12pt}{12pt}\textcolor{title}{\textsc{Undead Army}}}\\
At 15th level, you have near mastery of the necrotic arts. Once per day, you may cast “Create Undead” as a sixth-level without expending a spell slot. You may choose to expend a 3rd, 4th, or 5th level spell slot when you cast this spell. If you do, treat this spell as if it was cast as a 7th, 8th, or 9th level spell respectively. Once this ability is used, it may not be used again until you have completed a long rest. \\
\\
{\fontsize{12pt}{12pt}\textcolor{title}{\textsc{God of Death}}}\\
At 20th level, you can assume the form of a fallen angel. Using your action, you undergo a transformation. For 1 hour you gain the following features: \\
\begin{itemize}
\setlength\itemsep{-6pt}
\item Black wings sprout from your back and grant you a flying speed of 60 feet.
\item Your melee weapon strikes and spells deal an additional 3d8 necrotic damage. This damage is increased to 4d8 necrotic damage against a target that is celestial or against an aberration.
\end{itemize}
Once you use this feature, you can't use it again until you finish a long rest.\\
\\
{\fontsize{14pt}{14pt}\textcolor{title}{\textsc{Vow of Retribution}}}
\HRule{1pt}{8pt}
An anti-paladin who has taken the Vow of Retribution seeks only to bring a god to justice for some wrong they have committed. While those on a path of justice might now always follow the law, they generally refer to themselves as good-natured and cause-driven. To bring a god to justice is a great feat, therefore most anti-paladins who take this vow dedicate their life to the cause.\\
\\
{\fontsize{12pt}{12pt}\textcolor{title}{\textsc{Tenets of Retribution}}}\\
\indent\textbf{Spread their Tales: }Weaken the gods by speaking of their crimes\\
\indent\textbf{To their Knees: }Capture the gods when they are weakened\\
\indent\textbf{Determin their Punishment: }State a sentence to fit their crimes\\
\indent\textbf{Deliver Justive: }Fulfill that sentence, be it death or removal of powers
\\
{\fontsize{12pt}{12pt}\textcolor{title}{\textsc{Vow Spells}}}\\
You gain vow spells at the anti-paladin level specified.\\
\\
\begin{tabular}{cl}
\begin{tabular}[c]{@{}c@{}}Anti-Paladin\\ Level\end{tabular} & \multicolumn{1}{c}{Spells}         \\
\rowcolor[HTML]{B0E1AF} 
3rd                & Ensnaring Strike, Hex              \\
11-20              & Ray of Enfeeblement, Zone of Truth \\
\rowcolor[HTML]{B0E1AF} 
21-30              & Bestow Curse, Magic Circle         \\
31-40              & Confusion, Phantasmal Killer       \\
\rowcolor[HTML]{B0E1AF} 
41-50              & Cloudkill, Contagion              
\end{tabular}
* Must select the necrotic damage option\\
\\
{\fontsize{12pt}{12pt}\textcolor{title}{\textsc{Twist Divinity}}}\\
When you take this vow at 3rd level, you gain access to the following two Twist Divinity options.\\
\indent\textbf{Unholy Circle: }As an action, expend your Twist Divinity to create a 10 ft. radius circle of unholy light centered on yourself for one minute. Within this range, any attack roll against a celestial or aberration has advantage, and any attack roll by a celestial or aberration has disadvantage. In addition, a Celestial or aberration has disadvantage on Int, Wis, and Cha saving throws. \\
\\
\indent\textbf{Turn the Holy: }As an action, you present your unholy symbol and speak a phrase cursing aberrations and celestials, expending your Twist Divinity. Each celestial or aberration within 30 ft. of you must make a Wis saving throw. If the creature fails its saving throw, it is turned for one minute or until it takes damage. A turned creature must spend its turns trying to move as far away from you as it can, and it can’t willingly move to a space within 30 feet of you. It also can’t take reactions. For its action, it can only take the Dash action or try to escape from an effect that prevents it from moving. If there is nowhere to move, the creature can take the Dodge action. \\
\\
{\fontsize{12pt}{12pt}\textcolor{title}{\textsc{Resist Divination}}}\\
At 7th level, you gain immunity to radiant damage. Additionally, you gain advantage on saving throws against divine spells (such as those cast by a cleric, paladin, or other holy source).\\
\\
{\fontsize{12pt}{12pt}\textcolor{title}{\textsc{Dread Circle}}}\\
By 15th level, your training in dark magic has reached a new level. As an action, you can slam your weapon to the ground to produce the effect of a “Circle of Death” spell centered on yourself. You and your allies are not affected by this spell.\\
\indent Once you use this ability, you can’t use it again until you finish a long rest. \\
\\
{\fontsize{12pt}{12pt}\textcolor{title}{\textsc{True Justicar}}}\\
At 20th level, you can take on the aspects of a god in order to fight one. For example, you may give off a visible aura, float a very short distance off the ground, gain a booming voice, etc. The effect or effects chosen do not change a creature mechanically.
\indent For one minute you gain the following featuers:\\
\begin{itemize}
\setlength\itemsep{-6pt}
\item A fly speed equal to your movement speed
\item At the start of each of your turns, you regain 10 hit points
\item Whenever you cas an anti-paladin spell that has a casting time of one action, you can cast it using a bonus action instead
\item Enemy creatures within 10 feet of you have disadvantage on saving throws against your anti-paladin spells and Twist Divinity options.
\end{itemize}
Once you use this feature, you can't use it again until you finish a long rest.
\end{multicols*}
\newpage
Credits:
Walrock() for background, and the guide that caused me to learn to do this in \TeX\\
Smyris for the collection of font's, brushes and outlines.\\
\end{document}