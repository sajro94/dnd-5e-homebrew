\documentclass[22pt,a4paper,twocolumn]{article}
\usepackage[utf8]{inputenc}
\usepackage{amsmath}
\usepackage{amsfonts}
\usepackage{amssymb}
\usepackage{color}
\usepackage{fontspec}
\usepackage[table,xcdraw]{xcolor}
\usepackage{wallpaper}
\definecolor{title}{RGB}{94,18,0}
\usepackage[top=4.2in, left=0.6in, right=0.6in, bottom=1in]{geometry}
\ULCornerWallPaper{1}{pageBackground.jpg}
\newfontfamily\bookmania{Bookmania}
\setmainfont{Bookmania}[Scale=0.9]
\begin{document}
\ThisULCornerWallPaper{1}{pageBackgroundAfterlife.jpg}
\noindent{\fontsize{26pt}{26pt}\textcolor{title}{\bookmania \textsc{Oath of Afterlife}}}\\
The Oath the Afterlife, is the oath to make sure that all creatures that are destined to die do so. And that once they are dead they are helped and guided to their rightful place in the Afterlife. These paladins often seems bloodthirsty as they very often is out to kill someone, yet once someone is dead they treat them with absolute respect.\\
\\
{\fontsize{18pt}{18pt}\textcolor{title}{\textsc{Tenets of the Afterlife}}}\\
\textbf{Noone lives forever: }Immortal creatures are abominations and should be killed so their soul can rest.\\
\textbf{Respect the dead: }Any dead body deserves respect, but their soul deserves it even more.\\
\textbf{Help the restless: }Just as immortals should be killed, so should restless souls be helped\\
\\
{\fontsize{18pt}{18pt}\textcolor{title}{\textsc{Oath Spells}}}\\
You gain the oath spells at the paladin levels listed\\
\\
{\fontsize{16pt}{16pt}\textsc{\textbf{Oath of Afterlife spells}}}\\
\begin{tabular}{cp{6cm}}
{\bf Paladin Level} & {\bf Spells}                         \\
\rowcolor[HTML]{B8EFAD} 
3rd                 & Protection from Evil and Good, Comprehend Languages  \\
5th                 & Gentle Repose, Aid \\
\rowcolor[HTML]{B8EFAD} 
9th                 & Speak with Dead, Remove Curse \\
13th                & Banishment, Phantasmal Killer  \\
\rowcolor[HTML]{B8EFAD} 
17th                & Dispel Evil and Good, Commune
\end{tabular}\\\\
\\
{\fontsize{18pt}{18pt}\textcolor{title}{\textsc{Channel Divinity}}}\\
When you take this oath at 3rd level, you gain the following two Channel Divinity options.\\
\indent\textbf{This is your fate: }
By spending an action you can give a single creature advantage or disadvantage on death saving throws for, this effect lasts until the creature stabilises or dies.\\
\indent\textbf{Guide the soul: }As a bonus action you gain advantage on all concentration checks and all saves for 1 minute. This ability requires a dead creature to be within 15 feet, and the creature have to have died within 1 hour.\\\vspace{2cm}
\\
{\fontsize{18pt}{18pt}\textsc{\textcolor{title}{Aura of Afterlife}}}\\
Every ally within 10 feet of you gains advantage against fear effect, as they know they have nothing to fear. Because if they die you will guide them safely and correctly through the afterlife.\\
At the same time the mere presence of you scares your enemies, and they gain disadvantage on saves against fear effects.\\
\indent At 18th level the range of this extends to 30 feet.\\
\\
{\fontsize{18pt}{18pt}\textsc{\textcolor{title}{Guide for the souls}}}\\
Any creature that is incapable of dying from old age, have disadvantage on any roll against you, whether it is attack, saving throws or skill checks.\\
\\
{\fontsize{18pt}{18pt}\textsc{\textcolor{title}{Grim Reaper}}}\\
At 20th level, you can assume the form of a hooded pale figure, exactly what people see is indescribable because everyone sees there own embodiment of death.\\
\indent Using your action you undergo a transformation.\\
For 1 minute you gain the following benefits.
\begin{itemize}
\item Any creature affected by you Guide for the souls ability takes 2d8 points of damage each round.
\item Your smite can deal both Necrotic or Radiant damage at your decision
\item You are immune to any spell of the necromancy sub school
\end{itemize}
\end{document}