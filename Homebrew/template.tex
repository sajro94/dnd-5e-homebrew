\documentclass[22pt,a4paper,twocolumn]{article}
\usepackage[utf8]{inputenc}
\usepackage{amsmath}
\usepackage{amsfonts}
\usepackage{amssymb}
\usepackage{color}
\usepackage{pdfpages}
\usepackage{fontspec}
\usepackage[table,xcdraw]{xcolor}
\usepackage{wallpaper}
\definecolor{title}{RGB}{94,18,0}
\usepackage[top=4.1in, left=0.6in, right=0.6in, bottom=1in]{geometry}%Change the top to 0.6 if you have no banner, or increase/decrease if your banner is of different height
\ULCornerWallPaper{1}{pageBackground.jpg}%This is the background used, this shouldn't be changed.
\newfontfamily\bookmania{Bookmania}
\setmainfont{Bookmania}[Scale=0.9]
\newcommand{\dndtitle}[2]{\noindent{\fontsize{#1}{#1}\textcolor{title}{\textsc{#2}}}\newline}
\begin{document}
\ThisULCornerWallPaper{1}{pageTopBanner.jpg}%This line is used to insert a nice top banner if you have one, the banner should have a full page background that matches the normal background. If you have no Banner remove this line.
\dndtitle{26pt}{<Type> Name}
This is used for the Archetype Name, Class Name, Background Name, Race Name or similar\\
\\
\dndtitle{18pt}{<Type> Feature}
This is used as the title for different types of features.\\
\\
\dndtitle{16pt}{Feature Option}
If a feature have options or subsections, this is what you would use there\\
\\
\dndtitle{14pt}{Other titles}
And this is for even smaller tittles such as a subsubsection sort of thing.\\
\\
\begin{tabular}{cp{6cm}}
{\bf Class Level} & {\bf Spells}                         \\
\rowcolor[HTML]{B8EFAD} 
3rd                 & Class Spell, Other Class's spell\\
5th                 & Class Spell, Other Class's spell\\
\rowcolor[HTML]{B8EFAD} 
9th                 & Class Spell, Other Class's spell\\
13th                & Class Spell, Other Class's spell\\
\rowcolor[HTML]{B8EFAD} 
17th                & Class Spell, Other Class's spell\\
\end{tabular}\\
This is an example of a spell table for additional spells, this can also be used for other types of tables. This is sized to be two columns and fit the width of 1 column
\end{document}